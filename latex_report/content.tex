
\section*{Введение}
\addcontentsline{toc}{section}{Введение}
\thispagestyle{plain}

Машина Тьюринга является одной из основополагающих концепций в теории вычислений, предложенной Аланом Тьюрингом в 1936 году. Она служит абстрактной моделью вычислений и позволяет исследовать границы того, что может быть вычислено.

Понимание принципов машины Тьюринга помогает глубже осознать работу алгоритмов.

\newpage

\section{Формулирование проблемы и возможного её решения}
\thispagestyle{plain}
\subsection{\textbf{Задача}}

Задача звучит так: реализовать машину Тьюринга с графическим интерфейсом, которая состоит из:

\begin{itemize}[noitemsep,nolistsep]
    \item   бесконечной в обе стороны ленты;
    \item   головки записи-чтения;
    \item   алфавита;
    \item   таблицы состояний.
\end{itemize}

В программе должна быть возможность загружать из файла и сохранять в файл.

\subsection{\textbf{План решения поставленной проблемы}}

Первым делом надо разобраться с основными возможностями машины Тьюринга, эти знания были получены на занятиях по данной дисциплине.

Хранение данных будет в файлах формата .json. А графический интерфейс будет выполнен с помощью библиотеки tkinter\cite{tkinter} языка Python\cite{python}.

Создание бесконечной в обе стороны ленты - самый интересный вопрос. Я решил использовать такую структуру данных, как двусвязный список (DoubleLinkedList)\cite{dll}. Именно на этой структуре данных основывается моя реализация машины Тьюринга.

После выбора "основы" машины, можно составить план реализации:

\begin{enumerate}[label=\arabic*., noitemsep,nolistsep]
    \item Определение формата состояний и правил;
    \item Определение формата данных, сохраненных в файл;
    \item Написание ленты на основе класса DoubleLinkedList;
    \item Добавление ленты в класс TuringMachine;
    \item Реализация взаимодействия ленты, алфавита, состояний и правил;
    \item Создание графического интерфейса.
\end{enumerate}



\section{Программная реализация алгоритма}

\subsection{\textbf{Использованные библиотеки}}
При создании кода я буду использовать следующие библиотеки: tkinter для графического интерфейса, json для работы с json файлами и typing для указания типов переменных в функциях.

\subsection{\textbf{Описание кода}}

\subsubsection{Лента}

Лента будет представлена двусвязным списком (DoubleLinkedList), в нем будет подкласс Node, то есть узел - ячейка на ленте, который будет иметь "магический" метод "str", который отвечает за представление ячейки.

\begin{lstlisting}
class DoubleLinkedList:
    class Node:
        def __init__(self, value=None, prev=None, next=None) -> None:
            self.value = value
            self.prev = prev
            self.next = next

        def __str__(self) -> str:
            return str(self.value)

    def __init__(self) -> None:
        self.head = None
        self.tail = None
        self.len = 0
\end{lstlisting}

Далее будут приведены основные методы, которые используется в реализации машины Тьюринга:

\textit{Добавление в начало и конец списка}. Эти методы помогают добавить слово на ленту, а так же добавлять символы при выходе за границы текущего слова:

\begin{lstlisting}
def push_front(self, value=None) -> None:
    node = self.Node(value=value)
    node.next = self.head
    if self.head is not None:
        self.head.prev = node
    if self.tail is None:
        self.tail = node
    self.head = node
    self.len += 1

def push_back(self, value=None) -> None:
    node = self.Node(value=value)
    node.prev = self.tail
    if self.head is None:
        self.head = node
    if self.tail is not None:
        self.tail.next = node
    self.tail = node
    self.len += 1
\end{lstlisting}


\textit{Получение какого-либо узла списка}. Необходимость данного метода объяснять не приходится, так как получить какой-либо узел требуется очень часто:

\begin{lstlisting}
def get_node(self, index: int) -> Node:
    node = self.head
    n = 0

    while n != index:
        if node is None:
            return node
        node = node.next
        n += 1

    return node

def __getitem__(self, index: int) -> Node:
    return self.get_node(index)
\end{lstlisting}


\textit{Очищение списка}. Очистить сразу весь список нужно в ситуации, когда пользователь загружает новую конфигурацию или просто нажимает кнопку сброса:

\begin{lstlisting}
def clear(self) -> None:
    while self.head is not None:
        self.pop_front()
\end{lstlisting}


\subsubsection{класс TuringMachine}

Класс TuringMachine содержит функции, представляющие машину Тьюринга без графического интерфейса.

\textit{Конструктор класса}. В нем создаются все необходимые поля для работы машины Тьюринга. И загружается слово на ленту:

\begin{lstlisting}
class TuringMachine:
    def __init__(self, path_to_file=None):
        self.tape = DoubleLinkedList()
        self.data = dict()
        self.path_to_file = path_to_file
        self.alphabet = ""
        self.word = ""
        self.rules = dict()
        self.states = set()
        self.currentCondition = None
        self.pointer = None

        self.__load_data(path_to_file=path_to_file)
        self.__set_data()
        self.__set_word()
\end{lstlisting}

\textit{Установка слова на ленту}:

\begin{lstlisting}
def __set_word(self) -> None:
    if len(self.word):
        for char in self.word:
            self.tape.push_back(char)
\end{lstlisting}

\textit{Получение данных из файла}. Полученные данные записываются в атрибут класса: словарь data:

\begin{lstlisting}
def __load_data(self, path_to_file) -> None:
    if path_to_file is not None:
        with open(path_to_file) as file:
            self.data = json.load(file)
\end{lstlisting}


\textit{Установка текущей конфигурации из словаря data}. Данные устанавливаются текущими значениями из словаря data:

\begin{lstlisting}
def __set_data(self) -> None:
    if len(self.data):
        self.alphabet = self.data["alphabet"]
        self.word = self.data["word"]
        self.load_rules()
        self.states = self.get_states()
        self.currentCondition = self.data["start_condition"]
        self.pointer = self.data["start_pointer"]
\end{lstlisting}


\textit{Выполнение шага}. С помощью этой функции происходит выполнение одного шага в машине Тьюринга.
В ней согласно текущему состоянию и символу на ленте, записывается новый символ на ленту, делается движение, а так же переход в новое состояние:

\begin{lstlisting}
def step(self) -> bool:
    self.currentRule = self.rules.get(
        self.currentCondition, self.rules.get(self.data["start_condition"], None)
    )
    symbol = self.tape[self.pointer].value
    command = self.currentRule["commands"][symbol].split()

    new_symbol, move, new_condition = command

    self.tape[self.pointer].value = new_symbol

    if move == "L":
        self.pointer -= 1
        if self.pointer < 0:
            self.tape.push_front("_")
            self.pointer = 0
    elif move == "R":
        self.pointer += 1
        if self.pointer > self.tape.len - 1:
            self.tape.push_back("_")

    if new_condition != "!":
        self.currentCondition = new_condition
    else:
        self.currentCondition = self.data.get("start_condition", "q0")
        return False

    return True
\end{lstlisting}


\subsubsection{GUI}

Код графического интерфейса есть в ~\hyperlink{word:tmGUI}{приложении}. В нем присутствуют функции выполняющие загрузку из файла, а так же сохранение текущей конфигурации в файл. Выполнение одного шага или полное выполнение программы доступно только после загрузки конфигурации. 


\subsubsection{Запуск}

Если происходит запуск именно файла app.py (а не импорт), то код исполняется:

\begin{lstlisting}
from tmGUI import *


if __name__ == "__main__":
    root = tk.Tk()
    gui = TuringMachineGUI(root)
    root.mainloop()
\end{lstlisting}


\section{Результат работы программы}

Представлено выполнение алгоритма бинарного инкремента:

\image{Images/start.jpg}{Старт программы. Загруженное слово на ленту}{fig:start}{1}

\image{Images/rules.jpg}{Список правил}{fig:rules}{1}

\image{Images/result.jpg}{Результат выполнения программы}{fig:result}{1}

\image{Images/save.jpg}{Сохранение конфигурации}{fig:save}{1}

\newpage


\section*{Заключение}
\addcontentsline{toc}{section}{Заключение}

Процесс реализации машины Тьюринга помог полностью разобраться с её особенностями. Также применение двусвязного списка для реализации ленты побудило повторить структуры данных. 

Конечную программу можно использовать для самостоятельной подготовки к семинарским занятиям по теме "Машина Тьюринга" для моделирования каких-то алгоритмов.
