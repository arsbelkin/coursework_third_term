\section*{Приложение 1} \label{application1}
\addcontentsline{toc}{section}{Приложение 1}
\pagestyle{empty}

\setcounter{figure}{0}

Программный код алгоритма:

\textit{tape.py:}

\begin{lstlisting}
from typing import Union


class DoubleLinkedList:
    class Node:
        def __init__(self, value=None, prev=None, next=None) -> None:
            self.value = value
            self.prev = prev
            self.next = next

        def __str__(self) -> str:
            return str(self.value)

    def __init__(self) -> None:
        self.head = None
        self.tail = None
        self.len = 0

    def is_empty(self) -> bool:
        return not self.__len

    def push_front(self, value=None) -> None:
        node = self.Node(value=value)
        node.next = self.head
        if self.head is not None:
            self.head.prev = node
        if self.tail is None:
            self.tail = node
        self.head = node
        self.len += 1

    def push_back(self, value=None) -> None:
        node = self.Node(value=value)
        node.prev = self.tail
        if self.head is None:
            self.head = node
        if self.tail is not None:
            self.tail.next = node
        self.tail = node
        self.len += 1

    def pop_front(self) -> Node:
        if self.head is None:
            return

        node = self.head
        if self.head.next is not None:
            self.head.next.prev = None
        else:
            self.tail = None
        self.head = self.head.next
        self.len -= 1

        return node

    def pop_back(self) -> Union[Node, None]:
        if self.tail is None:
            return

        node = self.tail
        if self.tail.prev is not None:
            self.tail.prev.next = None
        else:
            self.head = None
        self.tail = self.tail.prev
        self.len -= 1

        return node

    def get_node(self, index: int) -> Node:
        node = self.head
        n = 0

        while n != index:
            if node is None:
                return node
            node = node.next
            n += 1

        return node

    def __getitem__(self, index: int) -> Node:
        return self.get_node(index)

    def insert(self, index: int, value) -> Node:
        right = self.get_node(index)
        if right is None:
            return self.push_back(value=value)

        left = right.prev
        if left is None:
            return self.push_front(value=value)

        node = self.Node(value=value)
        node.prev = left
        node.next = right
        left.next = node
        right.prev = node

        return node

    def erase(self, index: int) -> Union[Node, None]:
        node = self.get_node(index)
        if node is None:
            return

        if node.prev is None:
            return self.pop_front()

        if node.next is None:
            return self.pop_back()

        left = node.prev
        right = node.next
        left.next = right
        right.prev = left

        return node

    def clear(self) -> None:
        while self.head is not None:
            self.pop_front()

    def print_list(self) -> None:
        current_node = self.head

        while current_node:
            print(current_node.value if current_node.value != "empty" else "_", end=" ")
            current_node = current_node.next
        print()

    def __str__(self):
        current_node = self.head
        result = ""
        while current_node:
            result += current_node.value
            current_node = current_node.next
        return result
\end{lstlisting}


\textit{turingMachine.py:}

\begin{lstlisting}
from tape import DoubleLinkedList
import json


class TuringMachine:
    def __init__(self, path_to_file=None):
        self.tape = DoubleLinkedList()
        self.data = dict()
        self.path_to_file = path_to_file
        self.alphabet = ""
        self.word = ""
        self.rules = dict()
        self.states = set()
        self.currentCondition = None
        self.pointer = None

        self.__load_data(path_to_file=path_to_file)
        self.__set_data()
        self.__set_word()

    def __load_data(self, path_to_file) -> None:
        if path_to_file is not None:
            with open(path_to_file) as file:
                self.data = json.load(file)

    def __set_data(self) -> None:
        if len(self.data):
            self.alphabet = self.data["alphabet"]
            self.word = self.data["word"]
            self.load_rules()
            self.states = self.get_states()
            self.currentCondition = self.data["start_condition"]
            self.pointer = self.data["start_pointer"]

    def __set_word(self) -> None:
        if len(self.word):
            for char in self.word:
                self.tape.push_back(char)

    def load_rules(self):
        if len(self.data):
            self.rules = self.data["conditions"]

    def save(self, file):
        self.data["alphabet"] = self.alphabet
        self.data["word"] = str(self.tape)
        json.dump(self.data, file)

    def get_rules(self) -> str:
        rules = ""
        if len(self.rules):
            for cond in self.rules.values():
                for key, value in cond["commands"].items():
                    rules += f"{cond["condition"]}, "
                    rules += f"{key}:"
                    rules += f"{value}\n"
        return rules

    def get_states(self) -> set:
        if len(self.rules):
            return (cond.get("condition", "") for cond in self.rules.values())
        else:
            return self.states

    def set_rules(self, new_rules: str) -> bool:
        self.data["conditions"] = {}

        for line in new_rules.split("\n"):
            parts = line.split(":")
            key = tuple(map(str.strip, parts[0].split(",")))
            value = tuple(map(str.strip, parts[1].split()))

            if "conditions" not in self.data:
                self.data["conditions"] = {}

            if key[0] not in self.data["conditions"]:
                self.data["conditions"][key[0]] = {}
                self.data["conditions"][key[0]]["condition"] = key[0]

            if "commands" not in self.data["conditions"][key[0]]:
                self.data["conditions"][key[0]]["commands"] = {}

            if key[1] not in self.data["conditions"][key[0]]["commands"]:
                self.data["conditions"][key[0]]["commands"][key[1]] = {}
            self.data["conditions"][key[0]]["commands"][
                key[1]
            ] = f"{value[0]} {value[1]} {value[2]}"

    def step(self) -> bool:
        self.currentRule = self.rules.get(
            self.currentCondition, self.rules.get(self.data["start_condition"], None)
        )
        symbol = self.tape[self.pointer].value
        command = self.currentRule["commands"][symbol].split()

        new_symbol, move, new_condition = command

        self.tape[self.pointer].value = new_symbol

        if move == "L":
            self.pointer -= 1
            if self.pointer < 0:
                self.tape.push_front("_")
                self.pointer = 0
        elif move == "R":
            self.pointer += 1
            if self.pointer > self.tape.len - 1:
                self.tape.push_back("_")

        if new_condition != "!":
            self.currentCondition = new_condition
        else:
            self.currentCondition = self.data.get("start_condition", "q0")
            return False

        return True

    def run(self) -> None:
        flag = True
        while flag:
            flag = self.step()
\end{lstlisting}


\hypertarget{word:tmGUI}{\textit{tmGUI.py:}}


\begin{lstlisting}
from turing_machine import TuringMachine
import tkinter as tk
from tkinter import messagebox, filedialog, ttk


class TuringMachineGUI:
    CELL_SIZE = 50  # Размер ячейки на ленте

    def __init__(self, master: tk.Tk):
        self.master = master
        self.master.title("Эмулятор машины Тьюринга")

        self.tm = TuringMachine()
        self.running = False
        self.flag = [False, False, False, False]

        self.create_interface()

    def create_interface(self):
        """Создаёт элементы интерфейса."""
        config_frame = tk.Frame(self.master)
        config_frame.pack(pady=10)

        # Лента
        self.canvas = tk.Canvas(self.master, height=100, bg="white")
        self.canvas.pack(fill=tk.X, pady=10)
        self.cells = []

        # Панель управления
        control_frame = tk.Frame(self.master)
        control_frame.pack(pady=10)

        self.load_button = tk.Button(
            control_frame,
            text="Загрузить конфигурацию",
            command=self.load_configuration,
        )
        self.load_button.pack(side=tk.LEFT, padx=5)

        self.settings_button = tk.Button(
            control_frame, text="Настройки", command=self.open_settings
        )
        self.settings_button.pack(side=tk.LEFT, padx=5)

        self.save_button = tk.Button(
            control_frame, text="Сохранить", state=tk.DISABLED, command=self.save
        )
        self.save_button.pack(side=tk.LEFT, padx=5)

        self.run_button = tk.Button(
            control_frame,
            text="Полный запуск",
            state=tk.DISABLED,
            command=self.run_machine,
        )
        self.run_button.pack(side=tk.LEFT, padx=5)

        self.stop_button = tk.Button(
            control_frame,
            text="Остановить",
            state=tk.DISABLED,
            command=self.stop_machine,
        )
        self.stop_button.pack(side=tk.LEFT, padx=5)

        self.step_button = tk.Button(
            control_frame, text="Шаг", state=tk.DISABLED, command=self.perform_step
        )
        self.step_button.pack(side=tk.LEFT, padx=5)

        self.reset_button = tk.Button(
            control_frame, text="Сброс", state=tk.DISABLED, command=self.reset_machine
        )
        self.reset_button.pack(side=tk.LEFT, padx=5)

        self.status_label = tk.Label(self.master, text="Статус: Ожидание конфигурации")
        self.status_label.pack()

    def update_ui_buttons(self, running=False, loaded=False):
        self.run_button.config(
            state=tk.NORMAL if loaded and not running else tk.DISABLED
        )
        self.step_button.config(
            state=tk.NORMAL if loaded and not running else tk.DISABLED
        )
        self.reset_button.config(state=tk.NORMAL if loaded else tk.DISABLED)
        self.stop_button.config(state=tk.NORMAL if running else tk.DISABLED)
        self.load_button.config(state=tk.DISABLED if running else tk.NORMAL)
        self.settings_button.config(state=tk.DISABLED if running else tk.NORMAL)
        self.save_button.config(state=tk.NORMAL if loaded else tk.DISABLED)

    def save(self):
        filename = filedialog.asksaveasfile(filetypes=[("JSON Files", "*.json")])
        if not filename:
            return
        self.tm.save(filename)
        filename.close()

    def open_settings(self):
        if not self.tm:
            self.tm = TuringMachine()

        settings_window = tk.Toplevel(self.master)
        settings_window.title("Настройки машины Тьюринга")
        settings_window.geometry("600x400")

        tab_control = ttk.Notebook(settings_window)


        alphabet_tab = ttk.Frame(tab_control)
        tab_control.add(alphabet_tab, text="Алфавит")

        tk.Label(alphabet_tab, text="Введите символы алфавита через запятую:").pack(
            pady=10
        )
        alphabet_entry = tk.Entry(alphabet_tab, width=50)
        alphabet_entry.pack(pady=5)
        alphabet_entry.insert(0, ",".join(self.tm.alphabet))

        def save_alphabet():
            new_alphabet = alphabet_entry.get().split(",")
            if new_alphabet:
                self.tm.alphabet = [s.strip() for s in new_alphabet if s.strip()]
                self.tm.data['alphabet'] = self.tm.alphabet
                self.flag[0] = True
                if all(self.flag):
                    self.update_ui_buttons(loaded=True)
                messagebox.showinfo("Успех", "Алфавит обновлён!")

        tk.Button(alphabet_tab, text="Сохранить", command=save_alphabet).pack(pady=10)


        word_tab = ttk.Frame(tab_control)
        tab_control.add(word_tab, text="Слово")

        tk.Label(word_tab, text="Введите начальное слово:").pack(pady=10)
        word_entry = tk.Entry(word_tab, width=50)
        word_entry.pack(pady=5)

        
        if self.tm.tape.head:
            word_entry.insert(0, "".join(str(self.tm.tape)))

        def save_word():
            new_word = word_entry.get().strip()

            
            if any((char not in self.tm.alphabet and char != "_") for char in new_word):
                messagebox.showerror(
                    "Ошибка", "В слово можно вводить только символы из алфавита!"
                )
                return

            if new_word:
                self.tm.tape.clear()
                for char in new_word:
                    self.tm.tape.push_back(char)
                self.update_tape_display()
                self.flag[1] = True
                if all(self.flag):
                    self.update_ui_buttons(loaded=True)
                messagebox.showinfo("Успех", "Слово обновлено!")

        tk.Button(word_tab, text="Сохранить", command=save_word).pack(pady=10)

        
        states_tab = ttk.Frame(tab_control)
        tab_control.add(states_tab, text="Состояния")

        tk.Label(states_tab, text=f"Введите состояния через запятую:").pack(pady=10)
        states_entry = tk.Entry(states_tab, width=50)
        states_entry.pack(pady=5)
        states_entry.insert(0, ",".join(self.tm.get_states()))

        tk.Label(states_tab, text="Начальное состояние:").pack(pady=10)
        start_state_entry = tk.Entry(states_tab, width=50)
        start_state_entry.pack(pady=5)
        start_state_entry.insert(0, self.tm.data.get("start_condition", ""))

        tk.Label(states_tab, text="Начальный указатель:").pack(pady=10)
        start_pointer_entry = tk.Entry(states_tab, width=50)
        start_pointer_entry.pack(pady=5)
        start_pointer_entry.insert(0, str(self.tm.data.get("start_pointer", "")))

        def save_states():
            new_states = states_entry.get().split(",")
            new_start_state = start_state_entry.get().strip()
            try:
                new_pointer_state = int(start_pointer_entry.get().strip())
                if self.tm.tape.len > new_pointer_state < 0:
                    raise ValueError
            except ValueError:
                messagebox.showinfo(
                    "Ошибка", "Начальный указатель должен быть целочисленным!"
                )

            if new_states and new_start_state and str(new_pointer_state):
                self.tm.states = [s.strip() for s in new_states if s.strip()]
                self.tm.data["start_pointer"] = new_pointer_state
                self.tm.pointer = new_pointer_state
                if new_start_state in self.tm.states:
                    self.tm.data["start_condition"] = new_start_state
                    self.flag[2] = True
                    if all(self.flag):
                        self.update_ui_buttons(loaded=True)
                    messagebox.showinfo("Успех", "Состояния обновлены!")
                else:
                    messagebox.showinfo(
                        "Ошибка", "Начальное состояние должно быть списка состояний!"
                    )
                self.update_tape_display()

        tk.Button(states_tab, text="Сохранить", command=save_states).pack(pady=10)

        # Вкладка "Правила"
        rules_tab = ttk.Frame(tab_control)
        tab_control.add(rules_tab, text="Правила")

        tk.Label(
            rules_tab,
            text="Введите правила переходов:\nВ формате: state, symbol:new_symbol move new_state",
        ).pack(pady=10)
        rules_text = tk.Text(rules_tab, height=10, width=60)
        rules_text.pack(pady=5)

        # Отображаем текущие правила
        rules_text.insert(1.0, self.tm.get_rules())

        def save_rules():
            new_rules = rules_text.get(1.0, tk.END).strip()
            try:
                self.tm.set_rules(new_rules=new_rules)
                self.tm.load_rules()
                self.flag[3] = True
                if all(self.flag):
                    self.update_ui_buttons(loaded=True)
                messagebox.showinfo("Успех", "Правила обновлены!")
            except Exception as e:
                messagebox.showerror("Ошибка", f"Некорректный формат правил: {str(e)}")

        tk.Button(rules_tab, text="Сохранить", command=save_rules).pack(pady=10)

        tab_control.pack(expand=1, fill="both")

    def load_configuration(self):
        """Загрузка конфигурации из файла JSON."""
        filename = filedialog.askopenfilename(filetypes=[("JSON Files", "*.json")])
        if not filename:
            return
        try:
            # Загружаем новую конфигурацию
            self.tm = TuringMachine(path_to_file=filename)
            self.flag = [True, True, True, True]
            self.update_tape_display()

            # Обновляем состояние интерфейса
            self.update_ui_buttons(loaded=True)
            self.status_label.config(text="Конфигурация загружена.")
            self.master.geometry(
                f"{max(self.tm.tape.len * self.CELL_SIZE + 100, 1000)}x250"
            )
        except Exception as e:
            messagebox.showerror(
                "Ошибка", f"Не удалось загрузить конфигурацию: {str(e)}"
            )

    def update_tape_display(self):
        """Обновление отображения ленты на экране."""
        if not self.tm:
            return
        self.canvas.delete("all")
        self.cells = []

        current_node = self.tm.tape.head
        index = 0
        while current_node:
            x_start = index * self.CELL_SIZE
            x_end = x_start + self.CELL_SIZE
            rect = self.canvas.create_rectangle(
                x_start, 20, x_end, 70, fill="white", outline="black"
            )
            text = self.canvas.create_text(
                (x_start + x_end) // 2, 45, text=current_node.value, font=("Arial", 14)
            )
            self.cells.append((rect, text))
            if index == self.tm.pointer:
                self.canvas.itemconfig(rect, fill="lightblue")
            current_node = current_node.next
            index += 1

    def perform_step(self):
        """Выполнение одного шага."""
        if not self.tm:
            return
        try:
            result = self.tm.step()
            self.update_tape_display()
            if not result:
                self.update_ui_buttons(loaded=True)
                messagebox.showinfo("Финал", "Работа завершена.")
        except Exception as e:
            messagebox.showerror("Ошибка", f"Произошла ошибка: {str(e)}")

    def run_machine(self):
        """Полный запуск машины."""
        if not self.tm:
            return
        self.running = True
        self.update_ui_buttons(running=True)

        def step_through():
            if not self.running:
                self.update_ui_buttons(loaded=True)
                return
            try:
                result = self.tm.step()
                self.update_tape_display()
                if not result:
                    self.running = False
                    self.update_ui_buttons(loaded=True)
                    messagebox.showinfo("Финал", "Работа завершена.")
                else:
                    self.master.after(500, step_through)
            except Exception as e:
                self.running = False
                self.update_ui_buttons(loaded=True)
                messagebox.showerror("Ошибка", f"Произошла ошибка: {str(e)}")

        step_through()

    def stop_machine(self):
        """Останавливает выполнение машины."""
        self.running = False
        self.status_label.config(text="Выполнение остановлено.")
        self.update_ui_buttons(loaded=True)

    def reset_machine(self):
        """Сброс машины к начальному состоянию."""
        if not self.tm:
            return
        try:
            self.tm = TuringMachine()
            self.update_tape_display()
            self.update_ui_buttons(loaded=False)
            self.status_label.config(text="Машина сброшена.")
            self.flag = [False, False, False, False]
        except Exception as e:
            messagebox.showerror("Ошибка", f"Не удалось сбросить машину: {str(e)}")
\end{lstlisting}
